A conteporary approach to simulating complex multi-agent systems relies on the concept of 
behavior-based systems (BBS) \cite{mataric-1992}. The core of the concept is the idea that an 
intelligent behavior of a group is an emergent phenomenon of multiple interactions of agents with 
one another and an environment they exist in, along with capability of those to generalize an 
experience. It implies that agents have a memory and cognitive capabilities of some 
sort. This approach enables one to model a wide spectrum of systems including complex ones.

Some modern agent-based modeling automation software (ABMAS) offer facitlities to incorporate 
feedback loops through which an agent would get updated on changes in the environment. However, neither this, nor other modelling environments provide necessary 
facilities for modeling more complex aspects of agent interactions such as making inferences from 
experience and applying those inferences' results to a reasoning process.

Simulation software lag the concept of BBS, and among multiple shapes of forms of multi-agent 
systems that reside within the, gap we highlight those with a coordinator. By mentioning the 
"coordinator", we imply some entity (probably, a special class of agent) that establishes high-level 
strategic goals for another agents from the group that is subjected to the coordinator's control. 
Simulations like that may model real-life complex systems such as organizational or social ones. In 
systems of that complexity, "agents" are bound to perform within the confinemets of some goal they 
try to achieve, but at the same time they are free to act as they see fit on a lower scale. The 
ultimate objective notwithstanding, they apply their own knowledge, reason, and situation awareness. 
The deviations from the strategy may be so substantial, so an agent's actions may seem to contradict 
the strategy itself while serving it in a way that is the best from the tactical standpoint. The 
reason is that no general directive can comprehend the entirety of a complex system, all its 
peculiarly intertwined components, multiple networks agents may form while performing tasks assigned 
to them, and many other aspects.

In this paper, we propose a practical approach to modeling the process of a group coordination based 
on using weighted preference graph and AHP algorithm \cite{saaty1990decision}. The approach may be 
implemented as a framework, or extension plugin to some ABMAS.

We have to stipulate that the idea to use AHP algorithm to model decision making in a 
multi-agent system is nothing new by this time \cite{cartvehishvili2018model, 
drakaki2018intelligent}, neither is dynamic changing of the preference graph's weights to  alter 
the behavior of agents \cite{zytniewski2016application, brintrup2010behaviour}. But first and foremost, 
in order to model group coordination, we offer to exploit the \textit{structure} of a preference 
hierarchy graph. Changes in its higher levels which we interpret as setting a strategic objective 
influences the entire decision making process, and we use that property for translating strategic 
objectives into smaller-scale tactical goals. Second, for modeling group coordination, we offer to 
use shared editing of the graph. To our best knowledge, those two features are unique, and therefore 
constitute novelty of our approach.

